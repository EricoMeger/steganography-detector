%%%%%%%%%%%%%%%%%%%%%%%%%%%%%%%%%%%%%%%%%%%%%%%%%%%%%%%%%%%%%%%%%%%%%%
% How to use writeLaTeX: 
%
% You edit the source code here on the left, and the preview on the
% right shows you the result within a few seconds.
%
% Bookmark this page and share the URL with your co-authors. They can
% edit at the same time!
%
% You can upload figures, bibliographies, custom classes and
% styles using the files menu.
%
%%%%%%%%%%%%%%%%%%%%%%%%%%%%%%%%%%%%%%%%%%%%%%%%%%%%%%%%%%%%%%%%%%%%%%

\documentclass[12pt]{article}

\usepackage{include/template/sbc-template}

\usepackage{graphicx,url}

%\usepackage[brazil]{babel}   
\usepackage[utf8]{inputenc}  

\renewcommand\refname{Referências} % Troca nome da secao de references para referencias

\sloppy

\title{Titulo}

\author{Érico Meger\inst{1}, Eros Henrique Lunardon Andrade\inst{1}, Guilherme Werneck de Oliveira\inst{1}}


\address{Campus Pinhais – Instituto Federal do Paraná (IFPR)
Pinhais - PR - Brasil}

\begin{document} 

\maketitle

\begin{abstract}
  This meta-paper describes the style to be used in articles and short papers
  for SBC conferences. For papers in English, you should add just an abstract
  while for the papers in Portuguese, we also ask for an abstract in
  Portuguese (``resumo''). In both cases, abstracts should not have more than
  10 lines and must be in the first page of the paper.
\end{abstract}
     
\begin{resumo} 
  Este meta-artigo descreve o estilo a ser usado na confecção de artigos e
  resumos de artigos para publicação nos anais das conferências organizadas
  pela SBC. É solicitada a escrita de resumo e abstract apenas para os artigos
  escritos em português. Artigos em inglês deverão apresentar apenas abstract.
  Nos dois casos, o autor deve tomar cuidado para que o resumo (e o abstract)
  não ultrapassem 10 linhas cada, sendo que ambos devem estar na primeira
  página do artigo \cite{AdamaszekAdamaszek10}.
\end{resumo}


\section{Introdução}

(\textit{Esteganografia é outro termo para comunicação encoberta. Ela opera por meio da ocultação de mensagens em objetos aparentemente comuns, 
os quais são então enviados ao destinatário pretendido. O requisito mais importante de qualquer sistema esteganográfico é que seja impossível 
para um observador externo distinguir entre objetos ordinários e aqueles que carregam dados secretos.}\footnote{Tradução nossa.}, \cite{fridrich2009})

A esteganografia, portanto, não se limita apenas ao ato de esconder informações, mas envolve um campo de estudo mais amplo que abrange técnicas, 
algoritmos e aplicações destinadas a garantir a confidencialidade e a discrição da comunicação. Em contraste com a criptografia, que protege 
o conteúdo das mensagens mas não esconde sua existência, a esteganografia busca mascarar o próprio ato de comunicação. Esse caráter sutil 
faz dela uma área estratégica tanto para aplicações legítimas — como proteção de direitos autorais, autenticação de documentos e privacidade 
de dados — quanto para usos maliciosos. Essa dualidade evidencia que a esteganografia não deve ser compreendida apenas sob uma ótica técnica, 
mas também dentro de um contexto social e político mais amplo.

Nesse sentido, ao longo da história e especialmente no cenário contemporâneo, observa-se o crescimento de mecanismos de repressão à liberdade 
de expressão e de comunicação por parte de diversos governos, que restringem o livre fluxo de ideias e discursos. Diante desse cenário, 
a esteganografia surge como uma alternativa tecnológica capaz de preservar a liberdade de expressão, fornecendo meios de comunicação seguros 
e discretos, capazes de escapar da vigilância e da censura. Essa relevância sociopolítica, aliada aos avanços técnicos, reforça a necessidade 
de um estudo aprofundado do tema, justificando a escolha deste trabalho como contribuição à compreensão de seus fundamentos, aplicações 
e implicações na sociedade atual.


\subsection{Objetivo}

-----

\section{Revisão bibliográfica} \label{sec:firstpage}

The first page must display the paper title, the name and address of the
authors, the abstract in English and ``resumo'' in Portuguese (``resumos'' are
required only for papers written in Portuguese). The title must be centered
over the whole page, in 16 point boldface font and with 12 points of space
before itself. Author names must be centered in 12 point font, bold, all of
them disposed in the same line, separated by commas and with 12 points of
space after the title. Addresses must be centered in 12 point font, also with
12 points of space after the authors' names. E-mail addresses should be
written using font Courier New, 10 point nominal size, with 6 points of space
before and 6 points of space after.

The abstract and ``resumo'' (if is the case) must be in 12 point Times font,
indented 0.8cm on both sides. The word \textbf{Abstract} and \textbf{Resumo},
should be written in boldface and must precede the text.

\section{Metodologia}

In some conferences, the papers are published on CD-ROM while only the
abstract is published in the printed Proceedings. In this case, authors are
invited to prepare two final versions of the paper. One, complete, to be
published on the CD and the other, containing only the first page, with
abstract and ``resumo'' (for papers in Portuguese).

\section{Resultados e discussões}

Section titles must be in boldface, 13pt, flush left. There should be an extra
12 pt of space before each title. Section numbering is optional. The first
paragraph of each section should not be indented, while the first lines of
subsequent paragraphs should be indented by 1.27 cm.

\subsection{Subsections}

The subsection titles must be in boldface, 12pt, flush left.

\section{Conclusão}\label{sec:figs}


Figure and table captions should be centered if less than one line
%(Figure~\ref{fig:exampleFig1}), otherwise justified and indented by 0.8cm on
both margins, as shown in Figure~%\ref{fig:exampleFig2}. The caption font must
be Helvetica, 10 point, boldface, with 6 points of space before and after each
caption.

% \begin{figure}[ht]
% \centering
% \includegraphics[width=.5\textwidth]{fig1.jpg}
% \caption{A typical figure}
% \label{fig:exampleFig1}
% \end{figure}

% \begin{figure}[ht]
% \centering
% \includegraphics[width=.3\textwidth]{fig2.jpg}
% \caption{This figure is an example of a figure caption taking more than one
%   line and justified considering margins mentioned in Section~\ref{sec:figs}.}
% \label{fig:exampleFig2}
% \end{figure}

In tables, try to avoid the use of colored or shaded backgrounds, and avoid
thick, doubled, or unnecessary framing lines. When reporting empirical data,
do not use more decimal digits than warranted by their precision and
reproducibility. Table caption must be placed before the table (see Table 1)
and the font used must also be Helvetica, 10 point, boldface, with 6 points of
space before and after each caption.

% \begin{table}[ht]
% \centering
% \caption{Variables to be considered on the evaluation of interaction
%   techniques}
% \label{tab:exTable1}
% \includegraphics[width=.7\textwidth]{table.jpg}
% \end{table}

\bibliographystyle{include/template/sbc}
\bibliography{include/bibliography/bibliography}

\end{document}