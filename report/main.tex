%%%%%%%%%%%%%%%%%%%%%%%%%%%%%%%%%%%%%%%%%%%%%%%%%%%%%%%%%%%%%%%%%%%%%%
% How to use writeLaTeX: 
%
% You edit the source code here on the left, and the preview on the
% right shows you the result within a few seconds.
%
% Bookmark this page and share the URL with your co-authors. They can
% edit at the same time!
%
% You can upload figures, bibliographies, custom classes and
% styles using the files menu.
%
%%%%%%%%%%%%%%%%%%%%%%%%%%%%%%%%%%%%%%%%%%%%%%%%%%%%%%%%%%%%%%%%%%%%%%

\documentclass[12pt]{article}

\usepackage{include/template/sbc-template}

\usepackage{graphicx,url}

%\usepackage[brazil]{babel}   
\usepackage[utf8]{inputenc}  

\renewcommand\refname{Referências} % Troca nome da secao de references para referencias

\sloppy

\title{Inteligência Artificial para Detecção de Esteganografia: Um Estudo com Ferramentas Open Source}

\author{Érico Meger\inst{1}, Eros Henrique Lunardon Andrade\inst{1}, Guilherme Werneck de Oliveira\inst{1}}


\address{Campus Pinhais – Instituto Federal do Paraná (IFPR)
Pinhais - PR - Brasil}

\begin{document}

\maketitle

\begin{abstract}
  This meta-paper describes the style to be used in articles and short papers
  for SBC conferences. For papers in English, you should add just an abstract
  while for the papers in Portuguese, we also ask for an abstract in
  Portuguese (``resumo''). In both cases, abstracts should not have more than
  10 lines and must be in the first page of the paper.
\end{abstract}

\begin{resumo}
  Este meta-artigo descreve o estilo a ser usado na confecção de artigos e
  resumos de artigos para publicação nos anais das conferências organizadas
  pela SBC. É solicitada a escrita de resumo e abstract apenas para os artigos
  escritos em português. Artigos em inglês deverão apresentar apenas abstract.
  Nos dois casos, o autor deve tomar cuidado para que o resumo (e o abstract)
  não ultrapassem 10 linhas cada, sendo que ambos devem estar na primeira
  página do artigo.
\end{resumo}

\section{Introdução}

O movimento do software livre se estabelece como um paradigma essencial para
promover transparência, colaboração e inovação no cenário tecnológico
contemporâneo. Segundo a Free Software Foundation, software livre é definido
pela sua capacidade de respeitar as liberdades e o controle dos usuários sobre
o software: a liberdade de executar o programa para qualquer propósito, de
estudá-lo e modificá-lo (acesso ao código-fonte é pré-requisito), de
redistribuir cópias e de distribuir versões modificadas para a comunidade,
conhecidas como as quatro liberdades essenciais \cite{gnu_freesw}.

Ao assegurar essas liberdades, o software livre não apenas fortalece a
confiança nas soluções digitais, por permitir auditoria e aprendizado mútuo,
mas também fomenta ambientes colaborativos dinâmicos, onde ferramentas podem
ser aprimoradas coletivamente. Essa filosofia de abertura e colaboração se
manifesta também no campo da inteligência artificial, por meio de bibliotecas
como PyTorch, TensorFlow e scikit-learn. Essas ferramentas de código aberto
democratizam o acesso a algoritmos de aprendizado de máquina, permitindo
reprodutibilidade científica, auditoria de modelos e desenvolvimento
colaborativo de soluções inovadoras \cite{pytorch_about, tensorflow_about}.

No contexto da esteganografia, a disponibilidade dessas bibliotecas open source
potencializa o avanço da área. A análise de imagens digitais, por exemplo, pode
se beneficiar de recursos de detecção de padrões e classificação automática
fornecidos por essas ferramentas, auxiliando tanto no desenvolvimento
de técnicas esteganográficas mais robustas quanto na criação de métodos de
detecção mais eficazes. Assim, a intersecção entre software livre, inteligência
artificial e esteganografia evidencia como a filosofia do código aberto não só
fortalece a confiança técnica, mas também amplia as possibilidades de pesquisa
e aplicação prática neste campo.

A esteganografia pode ser compreendida como uma técnica utilizada para esconder
informações em meios aparentemente comuns, de forma que um observador externo
não consiga identificar a presença de dados ocultos \cite{Fridrich2010}.

Essa área de estudo, portanto, não se limita apenas ao ato de esconder
informações, mas constitui um campo de estudo mais amplo que abrange técnicas,
algoritmos e aplicações destinadas a garantir a confidencialidade e a discrição
da comunicação. Em contraste com a criptografia, que protege o conteúdo das
mensagens mas não oculta sua existência, a esteganografia busca mascarar o
próprio ato de comunicação \cite{Fridrich2010}. Essa característica a torna uma
área estratégica tanto para aplicações legítimas, como autenticação de
documentos e proteção da privacidade, quanto para usos maliciosos. Tal
dualidade evidencia que a esteganografia deve ser compreendida não apenas sob
uma perspectiva técnica, mas também dentro de um contexto social e político
mais amplo.

Nesse sentido, ao longo da história, e de forma ainda mais acentuada no cenário
contemporâneo, observa-se o fortalecimento de mecanismos de vigilância e
controle sobre a comunicação digital. Na Europa, por exemplo, esse movimento se
materializa tanto em iniciativas de remoção massiva de conteúdos, com mais de
41 milhões de postagens bloqueadas apenas no primeiro semestre de 2025
\cite{poder3602025}, quanto em pressões políticas para enfraquecer a segurança
criptográfica, como a exigência de um backdoor no iCloud, que levou a Apple a
retirar a opção de criptografia de ponta a ponta de seus serviços no Reino
Unido \cite{guardian2025}. Embora tais medidas sejam frequentemente
justificadas em nome da segurança pública, a ausência de transparência sobre os
critérios de censura e o impacto direto na privacidade digital levantam sérias
preocupações. Nesse contexto, a esteganografia age como uma alternativa
tecnológica de resistência, capaz de proporcionar meios de comunicação
discretos e seguros, reforçando sua relevância sociopolítica e justificando o
aprofundamento de seu estudo.

\subsection{Objetivo}

Este trabalho tem como objetivo desenvolver e treinar modelos de inteligência
artificial para a detecção de esteganografia em imagens digitais, utilizando
bibliotecas de software livre.

\section{Revisão bibliográfica} \label{sec:firstpage}

\section{Metodologia}

\section{Resultados e discussões}

\section{Conclusão}\label{sec:figs}

% \begin{figure}[ht]
% \centering
% \includegraphics[width=.5\textwidth]{fig1.jpg}
% \caption{A typical figure}
% \label{fig:exampleFig1}
% \end{figure}

% \begin{figure}[ht]
% \centering
% \includegraphics[width=.3\textwidth]{fig2.jpg}
% \caption{This figure is an example of a figure caption taking more than one
%   line and justified considering margins mentioned in Section~\ref{sec:figs}.}
% \label{fig:exampleFig2}
% \end{figure}

% \begin{table}[ht]
% \centering
% \caption{Variables to be considered on the evaluation of interaction
%   techniques}
% \label{tab:exTable1}
% \includegraphics[width=.7\textwidth]{table.jpg}
% \end{table}

\bibliographystyle{include/template/sbc}
\bibliography{include/bibliography/bibliography}

\end{document}